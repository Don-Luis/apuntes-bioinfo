\documentclass[nochap]{config/ejercicios}

\title{Ejercicios y problemas (ALGBIO)}
\author{Sandra Mingo Ramírez}
\date{2024/25}

\usepackage[all]{nowidow}
\usepackage{listing}
\usepackage{color}
\usepackage{tabularx}

\definecolor{dkgreen}{rgb}{0,0.6,0}
\definecolor{gray}{rgb}{0.5,0.5,0.5}
\definecolor{mauve}{rgb}{0.58,0,0.82}

\lstset{frame=tb,
  language=Python,
  aboveskip=3mm,
  belowskip=3mm,
  showstringspaces=false,
  columns=flexible,
  basicstyle={\small\ttfamily},
  numbers=none,
  numberstyle=\tiny\color{gray},
  keywordstyle=\color{blue},
  commentstyle=\color{dkgreen},
  stringstyle=\color{mauve},
  breaklines=true,
  breakatwhitespace=true,
  tabsize=3
}

\begin{document}
\maketitle

\begin{problemS}
Sort the following functions according to their growth: \\
$n, \sqrt{n}, n^{1.5}, n^2, n \log n, n \log \log n, n \log^2 n, 2/n, 2^n, 2^{n/2}, 37, n^2 \log n, n^3$

Para ordenar las funciones utilizando la notación O, hay que ver qué ocurre cuando n tiende a infinito. De forma ascendente: \\
$2/n, 37,\sqrt{n}, n, n \log \log n, n \log n, n \log^2 n, n^{1.5}, n^2, n^2 \log n, n^3, 2^{n/2}, 2^n$
\end{problemS}

\begin{problemS} \textbf{Python}
A classical example of a Divide and Conquer algorithm is binary search over ordered tables, which is essentially done according to the following pseudocode:
\begin{lstlisting}
def bin_search(key, l_ints):
	m = len(l_ints)//2
	if key == l_ints[m]:
		return m
	elif key < l_ints[m]:
		search key on l_ints ip to index m-1 # left table search
	else:
		search key on l_ints from index m+1 # right table search
\end{lstlisting}

Expand the pseudocode into a correct recursive Python function.
\begin{lstlisting}
def bin_search(key, l_ints):
	m = len(l_ints)//2
	if key == l_ints[m]:
		return m
	elif key < l_ints[m]:
		bin_search(key, l_ints[:m]
	else:
		bin_search(key, l_ints[m + 1:]
\end{lstlisting}
\end{problemS}

\begin{problemS}
Identify a proper key operation for \texttt{bin\_search}. How many key comparisons are performed at most on the table [1, 2, 3, 4, 5, 6, 7] in successful searches? And in unsuccessful ones?

La operación clave es la comparación \texttt{key == l\_ints}. En el mejor de los casos, solo se ejecuta una vez, pero en el peor de los casos, 3 veces.
\end{problemS}

\end{document}
