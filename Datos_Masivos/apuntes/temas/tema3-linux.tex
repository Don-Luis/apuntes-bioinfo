%29/10 - Gonzalo
\chapter{La línea de comandos de Linux}
\section{Introducción}
\subsection{Test inicial}
La línea de comandos de Linux se puede abrir mediante la combinación de teclas Ctrl + Alt + t. Ejecutamos el siguiente comando: \texttt{wget archive.ics.uci.edu/ml/machine-learning-databases/adult/adult.data}. El comando wget sirve para la descarga no interactiva de ficheros desde la web, soportando los protocolos HTTP, HTTPS y FTP. Con dicho comando, nos decargamos desde la web el fichero adult.data que tiene un tamaño de 3,8 M con permisos de lectura y escritura para el usuario y grupo y solo de lectura para otros. Este fichero contiene datos demográficos de distintas personas, incluyendo la edad, empleo, educación, estado civil, etnia, sexo y país.
\begin{lstlisting}[language=bash]
#Valor del cuarto campo de la tercera fila
cut -d "," -f 4 adult.data | head -n 3 | tail -n 1 #HS-grad

#Cuántas líneas contienen Portugal
grep -c "Portugal" adult.data #37

#Línea en la que aparece Portugal por primera vez
grep -n "Portugal" adult.data | head -n 1 #360

#Valor del primer campo en la fila 1000 empezando por el final
cut -d "," -f 1 adult.data | tail -n 1000 | head -n 1 #40

#Contar el número de filas del fichero
wc -l adult.data #32562

#Crear un fichero con las primeras 1000 filas 
head -n 1000 adult.data > adult-first-1000.data

#Contar bytes del fichero nuevo
wc -c adult-first-1000.data #121895

#Contar las líneas que contienen el string "Married-civ-spouse"
grep -c "Married-civ-spouse" adult.data #14976

#Valor mayor y menor del primer campo
cut -d "," -f 1 adult.data | sort -n | tail -n 1 #90 
cut -d "," -f 1 adult.data | sort -n | head -n 2 #17, hay un blanco
\end{lstlisting}

\subsection{Redirección y tuberías}
Se puede redireccionar la salida a un fichero mediante > y >> para sobreescribirlo o añadirlo respectivamente. 

Con algunas acciones, la línea de comandos se queda bloqueada hasta que termine la tarea (por ejemplo, con gedit). Con Ctrl C se cierra el procedimiento que estaba corriendo y se recupera el control de la consola de comandos, pero los cambios sin guardar se pierden. Por tanto, se puede enviar ese trabajo al background mediante \&. Otra opción es Ctrl Z, que pausa el procedimiento y permite recuperar el control de la consola.

Las tuberías permiten redireccionar la salida de un comando como input de otro sin necesidad de crear ficheros. Estas tuberías pueden ser concatenadas.

\subsection{Filtros}
Un filtro es un programa que lee la entrada estándar, realiza alguna operación sobre ella y saca el resultado por la salida estándar. Normalmente, se combinan con tuberías, y algunos comandos que pueden servir como filtros son \texttt{head, tail, tr, fmt, grep, sort}.

El comando \texttt{tr} \marginpar[\footnotesize tr] \ sirve para traducir o reemplazar. Por ejemplo, \texttt{tr '[0-9]' '*'} reemplaza todos los números por asteriscos. Si se utiliza -d, se borra el parámetro que se le dé en lugar de reemplazarlo. Asimismo, -c sirve para reemplazar todo lo que no entre en el filtro a lo que se le pase como segundo parámetro. Otro argumento opcional es -s, que comprime  una serie de caracteres repetidos.

Los comandos \texttt{head} y \texttt{tail} \marginpar[\footnotesize head \\ tail] \ permiten mostrar las primeras y últimas líneas respectivamente, pero también tienen algunas opciones adicionales. Cuando se utiliza \texttt{-n -100}, se excluyen las últimas 100 líneas, mientras que \texttt{-n +100} hace que empiece en la fila 100. 

El comando \texttt{fmt} \marginpar[\footnotesize fmt] \ permite formatear ficheros para que las líneas sean más pequeñas, recolocar palabras, etc. Por ejemplo, \texttt{fmt -70 -s} hace que cada línea muestre como máximo 70 caracteres, mostrando los caracteres restantes en una nueva línea.

\begin{lstlisting}[language=bash]
#Muestra los 10 procesos más recientes del sistema del usuario actual
ps -ef | tr -s ' ' | grep 'sandra ' | tail -n 10

#Muestra las páginas del manual de grep con 100 caracteres por línea, sustituyendo los dígitos por asteriscos y redireccionando a un fichero
man grep | fmt -100 | tr '[0-9] '*' > file.txt
\end{lstlisting}

El comando \texttt{split} \marginpar[\footnotesize split] \  divide un fichero en distintos bloques de 1000 líneas de forma predeterminada. Ese valor se puede modificar con -l, o se puede definir la cantidad de ficheros de tamaño equitativo sin romper líneas con -n.

El comando \texttt{cut} \marginpar[\footnotesize cut] \ borra secciones de cada línea de un fichero. Con -d se define el delimitador de las distintas columnas, y con -f se selecciona la columna.

El comando \texttt{paste} \marginpar[\footnotesize paste] \ permite combinar varios ficheros que consistan de distintas columnas. Se puede definir el delimitador con -d, que de forma determinada utiliza tabulador. Así, es el comando contrario a cut. 

\begin{lstlisting}[language=bash]
#Extrae las columnas 6 y 4 del fichero adult.data en ese orden.
cat adult.data | cut -d ',' -f 6 > adult-f6.data
cat adult.data | cut -d ',' -f 4 > adult-f4.data
paste -d ',' adult-f6.data adult-f4.data > adult-f6-f4.data
\end{lstlisting}