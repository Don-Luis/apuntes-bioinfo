%29/10 - Gonzalo
\chapter{La línea de comandos de Linux}
\section{Introducción}
\subsection{Test inicial}
La línea de comandos de Linux se puede abrir mediante la combinación de teclas Ctrl + Alt + t. Ejecutamos el siguiente comando: \texttt{wget archive.ics.uci.edu/ml/machine-learning-databases/adult/adult.data}. El comando wget sirve para la descarga no interactiva de ficheros desde la web, soportando los protocolos HTTP, HTTPS y FTP. Con dicho comando, nos decargamos desde la web el fichero adult.data que tiene un tamaño de 3,8 M con permisos de lectura y escritura para el usuario y grupo y solo de lectura para otros. Este fichero contiene datos demográficos de distintas personas, incluyendo la edad, empleo, educación, estado civil, etnia, sexo y país.
\begin{lstlisting}[language=bash]
#Valor del cuarto campo de la tercera fila
cut -d "," -f 4 adult.data | head -n 3 | tail -n 1 #HS-grad

#Cuántas líneas contienen Portugal
grep -c "Portugal" adult.data #37

#Línea en la que aparece Portugal por primera vez
grep -n "Portugal" adult.data | head -n 1 #360

#Valor del primer campo en la fila 1000 empezando por el final
cut -d "," -f 1 adult.data | tail -n 1000 | head -n 1 #40

#Contar el número de filas del fichero
wc -l adult.data #32562

#Crear un fichero con las primeras 1000 filas 
head -n 1000 adult.data > adult-first-1000.data

#Contar bytes del fichero nuevo
wc -c adult-first-1000.data #121895

#Contar las líneas que contienen el string "Married-civ-spouse"
grep -c "Married-civ-spouse" adult.data #14976

#Valor mayor y menor del primer campo
cut -d "," -f 1 adult.data | sort -n | tail -n 1 #90 
cut -d "," -f 1 adult.data | sort -n | head -n 2 #17, hay un blanco
\end{lstlisting}

\subsection{Redirección y tuberías}
Se puede redireccionar la salida a un fichero mediante > y >> para sobreescribirlo o añadirlo respectivamente. 

Con algunas acciones, la línea de comandos se queda bloqueada hasta que termine la tarea (por ejemplo, con gedit). Con Ctrl C se cierra el procedimiento que estaba corriendo y se recupera el control de la consola de comandos, pero los cambios sin guardar se pierden. Por tanto, se puede enviar ese trabajo al background mediante \&. Otra opción es Ctrl Z, que pausa el procedimiento y permite recuperar el control de la consola.

Las tuberías permiten redireccionar la salida de un comando como input de otro sin necesidad de crear ficheros. Estas tuberías pueden ser concatenadas.

\subsection{Filtros}
Un filtro es un programa que lee la entrada estándar, realiza alguna operación sobre ella y saca el resultado por la salida estándar. Normalmente, se combinan con tuberías, y algunos comandos que pueden servir como filtros son \texttt{head, tail, tr, fmt, grep, sort}.

El comando \texttt{tr} \marginpar[\footnotesize tr] \ sirve para traducir o reemplazar. Por ejemplo, \texttt{tr '[0-9]' '*'} reemplaza todos los números por asteriscos. Si se utiliza -d, se borra el parámetro que se le dé en lugar de reemplazarlo. Asimismo, -c sirve para reemplazar todo lo que no entre en el filtro a lo que se le pase como segundo parámetro. Otro argumento opcional es -s, que comprime  una serie de caracteres repetidos.

Los comandos \texttt{head} y \texttt{tail} \marginpar[\footnotesize head \\ tail] \ permiten mostrar las primeras y últimas líneas respectivamente, pero también tienen algunas opciones adicionales. Cuando se utiliza \texttt{-n -100}, se excluyen las últimas 100 líneas, mientras que \texttt{-n +100} hace que empiece en la fila 100. 

El comando \texttt{fmt} \marginpar[\footnotesize fmt] \ permite formatear ficheros para que las líneas sean más pequeñas, recolocar palabras, etc. Por ejemplo, \texttt{fmt -70 -s} hace que cada línea muestre como máximo 70 caracteres, mostrando los caracteres restantes en una nueva línea.
\begin{lstlisting}[language=bash]
#Muestra los 10 procesos más recientes del sistema del usuario actual
ps -ef | tr -s ' ' | grep 'sandra ' | tail -n 10

#Muestra las páginas del manual de grep con 100 caracteres por línea, sustituyendo los dígitos por asteriscos y redireccionando a un fichero
man grep | fmt -100 | tr '[0-9] '*' > file.txt
\end{lstlisting}

El comando \texttt{split} \marginpar[\footnotesize split] \  divide un fichero en distintos bloques de 1000 líneas de forma predeterminada. Ese valor se puede modificar con -l, o se puede definir la cantidad de ficheros de tamaño equitativo sin romper líneas con -n.

El comando \texttt{cut} \marginpar[\footnotesize cut] \ borra secciones de cada línea de un fichero. Con -d se define el delimitador de las distintas columnas, y con -f se selecciona la columna.

El comando \texttt{paste} \marginpar[\footnotesize paste] \ permite combinar varios ficheros que consistan de distintas columnas. Se puede definir el delimitador con -d, que de forma determinada utiliza tabulador. Así, es el comando contrario a cut. 

\begin{lstlisting}[language=bash]
#Extrae las columnas 6 y 4 del fichero adult.data en ese orden.
cat adult.data | cut -d ',' -f 6 > adult-f6.data
cat adult.data | cut -d ',' -f 4 > adult-f4.data
paste -d ',' adult-f6.data adult-f4.data > adult-f6-f4.data
\end{lstlisting}

%31/10 - Gonzalo
Esto puede ser útil para combinar un fichero sin numerar con una columna que contenga la numeración.
\begin{lstlisting}[language=bash]
wc -l adult.data #32562
echo {1..32562} | tr ' ' '\n' > column-numbers.txt
paste -d ',' column-numbers.txt adult.data > adult-numbered.data
\end{lstlisting}

El comando \texttt{shuf} \marginpar[\footnotesize shuf] \  (de shuffle) permite generar permutaciones aleatorias, mezclando o barajando las líneas. Con el parámetro -r, se permiten las repeticiones, y -n permite especificar el número de mezclas.
\begin{lstlisting}[language=bash]
#Mezclar las primeras 10 filas
head adult-numbered.data | shuf
#Mezclar las primeras 5 filas
head adult-numbered.data | shuf -n 5
#Generar 15 filas mezcladas con repetición
head adult-numbered.data | shuf -n 15 -r
#Crear una permutación aleatoria de los números 2 a 9
shuf -i 2-9
\end{lstlisting}
 
 El comando \texttt{sort} \marginpar[\footnotesize sort] \ ordena las filas utilizando ordenación alfabética ASCII (las mayúsculas están por delante de las minúsculas; Z va antes que a). Para ignorar las mayúsculas y minúsculas, se debe poner -f, y para ordenación numérica, -n o -g. 
 
 El comando \texttt{uniq} \marginpar[\footnotesize uniq] \ elimina las filas adyacentes repetidas. Esto se suele utilizar junto a sort para primero ordenar las filas y luego borrar las repetidas. -c indica cuántas filas había de cada tipo. 
\begin{lstlisting}[language=bash]
#Extraer el tipo de región de un fichero GFF y contar cuántas filas hay de cada tipo
cut -f 3 GCF_000002765.3_ASM276v1_genomic.gff | grep -v '#' | sort | uniq -c
#Extraer una lista ordenada de todos los posibles valores del campo 6 de adult.data
cut -f 6 -d , adult.data | sort | uniq -c
\end{lstlisting}

El comando \texttt{join} \marginpar[\footnotesize join] \ une líneas de dos ficheros en base a un campo en común. -1 sirve para especificar el campo del fichero 1, y -2 igual para el segundo fichero. -t permite especificar el delimitador.
\begin{lstlisting}[language=bash]
cut adult-numbered.data -d ',' -f 1,3 | shuf -n 1000 | sort -t ',' -k 1,1 > f13.txt
cut adult-numbered.data -d ',' -f 1,7 | shuf -n 1000 | sort -t ',' -k 1,1 > f17.txt
join -t ',' -1 1 -2 1 f13.txt f17.txt
#Y lo mismo pero manteniendo todas las líneas del primer fichero
join -t ',' -1 1 -2 1 -a 1 f13.txt f17.txt
\end{lstlisting}

El comando \texttt{comm} \marginpar[\footnotesize comm] \ sirve para comparar dos ficheros ordenados de forma alfabética. El resultado son tres columnas. En la primera, se muestran los datos únicos del primer fichero, en la segunda, los datos únicos del segundo, y en la tercera, los datos comunes en ambos ficheros. 

\begin{lstlisting}[language=bash]
#Crea un fichero a.txt con 20 números aleatorios entre 0 y 100 sin repetir; cada número debe estar en una línea distinta
shuf -i 0-100 -n 20 | sort > a.txt
#Crea un segundo fichero b.txt con 20 números aleatorios entre 0 y 100 sin repetir; cada número debe estar en una línea distinta
shuf -i 0-100 -n 20 | sort > b.txt
#Muestra los números que aparezcan en ambos ficheros
join -t ',' -1 1 -2 2 a.txt b.txt
comm -12 a.txt b.txt
\end{lstlisting}

\subsection{Scripts de shell}
Todos los comandos vistos anteriormente se pueden utilizar en scripts de bash. Los script deben empezar con el intérprete.
\begin{lstlisting}[language=bash]
#!/bin/bash
#Hello world script

echo "Hello World!"
\end{lstlisting}

Para ejecutar un script, primero se le debe dar permisos de ejecución mediante \texttt{chmod +x hello-world.sh}. Dentro de un fichero se pueden crear variables asignando el valor con el igual sin dejar espacios. Para luego poder llamar a la variable, se debe llamar con un \$ delante del nombre.

\begin{lstlisting}[language=bash]
#!/bin/bash
#Variables:
input_file='adult.data'
output_file='adult5-7.data'
fields='5-7'

#Remove spaces and get selected fields:
cat $input_file | tr -d ' ' | cut -d ',' -f $fields > tmp.txt

#Sort and get unique lines:
sort tmp.txt | uniq > $output_file

#Remove temporary file:
rm tmp.txt
\end{lstlisting}

También es posible introducir argumentos a un script, los cuales se llaman con \$1-9. \$0 representa el nombre del fichero.
\begin{lstlisting}[language=bash]
#!/bin/bash
#Remove spaces and get selected fields:
cat $1 | tr -d ' ' | cut -d ',' -f $3 > tmp.txt

#Sort and get unique lines:
sort tmp.txt | uniq > $2

#Remove temporary file:
rm tmp.txt

#Show head of new file:
head -n 20 $2
\end{lstlisting}

Ejercicio: escribir un script que acepte tres argumentos: f, a, b. El argumento f es el nombre de un fichero csv, y los argumentos a y b son números integer. El script debe mostrar la lista de valores coincidentes en las columnas a y b del fichero f.
\begin{lstlisting}[language=bash]
#!/bin/bash
file=$1
column1=$2
column2=$3

cut -d ',' -f $column1 $file |sort > tmp1.txt
cut -d ',' -f $column2 $file | sort > tmp2.txt
join -t ',' -1 1 -2 1 tmp1.txt tmp2.txt

rm tmp1.txt tmp2.txt
\end{lstlisting}

\subsection{Editores de texto}
\texttt{vi} es un editor de texto que viene en todas las distribuciones de Linux. La ventaja es que no abre ventanas externas (como gedit). Tiene dos modos: modo de inserción y modo de comando. En modo comando, los caracteres que se pulsan se interpretan como comandos, y se puede navegar mediante hjkl. Pulsando i, se cambia al modo de inserción, pudiendo añadir texto al editor. Para salir sin guardar, se debe poner \texttt{:q!}, y para salir guardando, \texttt{:x}. Para volver al modo de comando desde el modo inserción, basta con pulsar Esc. 

\section{Expresiones regulares y grep}
El comando \texttt{grep} permite buscar texto, pero también patrones. Los patrones emplean distintos símbolos:
\begin{itemize}
\item $\wedge$ : inicio (por ejemplo, "$\wedge$Hola")
\item \$: final (por ejemplo, "Pepito\$")
\item .: comodín, cualquier carácter (por ejemplo, "Pepit." encuentra tanto Pepito como Pepita).
\item \lbrack \rbrack: indican un conjunto de caracteres válidos ("Pepit[ao]" encuentra Pepito y Pepita, pero no Pepite).
\item \lbrack  $\wedge$ \rbrack : en la posición de los corchetes, vale cualquier carácter menos los que se especifiquen dentro ("Pepit[$\wedge$o]" solo encuentra Pepita).
\item *: indica un número indeterminado de repeticiones del carácter anterior, siendo 0 una posibilidad. Por ejemplo, con la expresión "[1-9][0-9]*th", se encontrarían 01th, 15th, 2th y 2333th. De 01th, solo se encontraría 1th con el patrón, ya que el 0 no está incluido en el primer conjunto. Así, tanto 01th como 2th tienen el dígito en el primer corchete, el segundo corchete se repite 0 veces y finaliza en th. En el caso de 2333th, el 2 se encontraría en el primer corchete, y los 3 en el segundo corchete, repitiéndose las veces necesarias. 
\end{itemize} 

Una expresión regular que extraiga las líneas que contengan solo números sería la siguiente:
\begin{lstlisting}[language=bash]
#! /bin/bash
grep "^[0-9]*[0-9 ,]*$" $1
\end{lstlisting}

%05/11 - Gonzalo
Para los símbolos especiales (como un punto o un circunflejo), se debe preceder el símbolo con una barra invertida para mostrar que el carácter es el que se quiere buscar, no por su significado en los patrones. Es decir, para buscar todas las líneas que empiecen con una letra mayúscula y terminen en un punto, se debe poner:
\begin{lstlisting}[language=bash]
#! /bin/bash
grep '^[A-Z].*\.$'
\end{lstlisting}

Y las líneas que empiecen con circunflejo:
\begin{lstlisting}[language=bash]
#! /bin/bash
grep '^\^' 
\end{lstlisting}

\subsection{Expresiones reguladas extendidas}
\subsubsection{Agrupamientos}
En algunas ocasiones se quieren agrupar expresiones. Esto se realiza con paréntesis, donde se especifica la cadena completa a buscar. Si queremos que los paréntesis se interpreten como delimitadores de grupos, se debe escapar de ellos con las barras invertidas. 
\begin{lstlisting}[language=bash]
#Buscar la cadena GTA seguida de cero o más repeticiones de GTA
grep 'GTA\(GTA\)*'
#Buscar la cadena GTA seguida de cero o más repeticiones de (GTA)
grep 'GTA(GTA)*'
\end{lstlisting}

La opción -E permite interpretar patrones más extensos, entendiendo directamente los paréntesis como delimitadores de grupos. Así, cuando escapemos de ellos, lo que se interpreta es que se busque el símbolo del paréntesis dentro del patrón.
\begin{lstlisting}[language=bash]
#Buscar la cadena GTA seguida de cero o más repeticiones de (GTA)
grep -E 'GTA\(GTA\)*'
#Buscar la cadena GTA seguida de cero o más repeticiones de GTA
grep -E 'GTA(GTA)*'
\end{lstlisting}

\subsubsection{Alternancia}
También se pueden utilizar tuberías que funcionan como "or". 
\begin{lstlisting}[language=bash]
grep -E 'GACC|TCAG'
grep -E '(AA|GG|TT|CC)(AA|GG|TT|CC)*'
\end{lstlisting}

\subsubsection{Cuantificadores}
El símbolo ? indica una o ninguna repetición, mientras que + indica una o más repeticiones.
\begin{lstlisting}[language=bash]
grep -E '(GACC)?' #0 o 1
grep -E '(GACC)+' #1 o más
grep -E '(GACC)*' #0 o más
\end{lstlisting}

También es posible especificar el número de veces que se repita una serie de caracteres:
\begin{lstlisting}[language=bash]
#La cadena se repite exactamente 3 veces
grep -E '(GACC){3}'
#La cadena se repite entre 3 y 5 veces.
grep -E '(GACC){3,5}'
\end{lstlisting}

Ejercicios: 
\begin{lstlisting}[language=bash]
#Escribir una expresión regular que encuentre cadenas sin T y con un número impar de G
grep -E '[^TG]*G([^GT]*G[^GT]*G)*[^GT]*'

#Escribir una expresión regular que encuentre direcciones de e-mail
grep -E ' ^[a-z] (a-z0-9_)*\.(a-z0-9_)*@[a-z]*\.(es|com)'
\end{lstlisting}

\section{Expansiones}
\subsection{Expansión aritmética}
Para hacer operaciones aritméticas en la consola, se debe poner \texttt{\$(( operación ))}. 
\begin{lstlisting}[language=bash]
echo $((4+3))
\end{lstlisting}

\subsection{Expansión de llaves}
Se pueden crear listas con llaves:
\begin{lstlisting}[language=bash]
{list}

#Imprimir la lista A B C
echo {A, B, C}

#Imprimir la lista horse home hope horoscope
echo ho{rs, m, p, roscop}e
\end{lstlisting}

Se pueden utilizar dos puntos que implican el rango entre los dos valores. Así, \texttt{{x..z}} genera la lista x y z. También se puede anidar:
\begin{lstlisting}[language=bash]
echo {z..a}
echo{1..9}
echo{X{1,2}, Y{3,4}}
echo{X,Y}{1..4}
\end{lstlisting}

\section{Sustitución de comandos}
Mediante la sustitución de comandos se puede utilizar la salida de un comando como expansión.
\begin{lstlisting}[language=bash]
$(comando)

#Imprimir con echo la salida de ls
echo $(ls)
#Crear directorios con los nombres de los valores únicos de la cuarta columna de adult.data
mkdir $(cut -d ',' -f 4 adult.data | sort | uniq)
\end{lstlisting}

\section{El comando \texttt{sed}}
Sed permite editar documentos de texto sin editor. Por ejemplo, \texttt{sed 's/dog/cat/g'} sustituye (s) cada aparición (g) de dog por cat. Esto lo saca por pantalla, pero se puede redireccionar a otro fichero. En el caso de \texttt{sed '10,20s/dog/cat/g'}, la sustitución se realiza solo de las líneas 10 a 20. Si se eliminase la g, en una línea con varias apariciones solo se reemplazaría la primera.

Los comandos básicos son d para borrar la línea actual, p para imprimir la línea actual y n. La línea puede ser un número, un rango o una expresión regular.
\begin{lstlisting}[language=bash]
sed '125s/dog/cat/' #sustituye solo la línea 125
sed '$s/dog/cat/' #sustituye la última línea del fichero
sed '1i\Header' #agrega el texto "Header" antes de la primera línea.
sed '2~2s/dog/cat/' #sustituye las líneas pares
sed '/^[A-Z]/s/dog/cat' #sustituye líneas que empiecen por mayúscula

sed '10,20d' #borra las líneas 10 a 20
sed '10,20!/s/dog/cat/' #reemplaza dog por cat en todas las líneas excepto de la 10 a la 20
\end{lstlisting}

También se pueden agrupar comandos.
\begin{lstlisting}[language=bash]
sed '/^$/d' #elimina todas las líneas que empiezan y acaban sin tener nada entre medias, es decir, las líneas en blanco
seq 1 20 | sed -n '3,10p' #imprime las filas 3 y 10 de la salida de seq
seq 1 20 | sed -n '/[26]/{s/1/5/;p}' #muestra las líneas que tengan un 2 o un 6, sustituyendo en ellas el 1 por el 5
\end{lstlisting}

\section{El comando \texttt{awk}}
Este comando sirve para procesar texto línea a línea de cada fichero (como Excel en línea de comandos). awk sigue unas normas en la forma patrón \{acción\}. 
\begin{lstlisting}[language=bash]
awk 'BEGIN {print "Hello world"}' #imprime Hello world
awk '/regexp/ { print $0 }' input-file #imprime las líneas que cumplan la expresión regular
awk '/regexp/' input-file #igual que antes; la acción predeterminada es la impresión
awk '/regexp/ {}' input-file #no hace nada
awk 'length($0) > 50' input-file #imprime las líneas que tienen más de 50 caracteres
\end{lstlisting}

El ciclo awk es leer el fichero de entrada línea por línea. Por cada línea, awk intenta el patrón para cada norma. Para cada patrón que coincida, awk produce la acción de la norma de ese patrón. Si se cumplen varios patrones, entonces se ejecutan varias acciones. De igual forma, si no se cumple ningún patrón, no se ejecuta ninguna acción.

Awk lee la entrada en unidades llamadas registros. Los registros están separados por un carácter llamado separador de registros. El separador de registros es, por defecto, el carácter de nueva línea, por lo que los registros suelen ser líneas. El separador de registros puede cambiarse asignando un valor diferente a la variable RS. Awk lleva la cuenta del número de registros que se han leído en la variable NR. 

Cada registro se divide en partes llamadas campos. Por defecto, los campos están separados por espacios en blanco (uno o más espacios, tabuladores o nuevas líneas). El separador de campos se puede cambiar asignando la variable FS, o en la línea de comandos con la opción -F. El número de campos de un registro se almacena en la variable NF Se hace referencia a los campos con el signo \$:
\begin{itemize}
\item \$0 es todo el registro
\item \$i, siendo i un número mayor que 0, es el campo número i.
\item \$NF es el último campo
\end{itemize}

En el siguiente código, se imprime la longitud de la línea más larga. Al compienzo se inicializa la variable max que se va actualizando según se van pasando las líneas del fichero.
\begin{lstlisting}[language=bash]
awk 'BEGIN { max = 0 }
	{ if (length($0) > max) max = length($0) }
	END { print max }' adult.data
\end{lstlisting}

Otro ejemplo: imprimir la suma de los tamaños de los ficheros del usuario sandra:
\begin{lstlisting}[language=bash]
ls -l | awk 'BEGIN { sum = 0 } 
	$3 == "sandra" { sum += $5 }
	END {print sum }'
\end{lstlisting}

OFS es el separador de campo de salida. Así, se puede leer un fichero separado por comas y que el fichero de salida esté separado por punto y coma, por ejemplo.
\begin{lstlisting}[language=bash]
#Se cambia el delimitador de espacio a dos puntos, el segundo valor (b) se reemplaza por un string vacío, se crea una sexta variable (nueva) con valor new, se imprime la línea y se imprime el número de registros.
echo a b c d | awk '{ OFS = ":"; $2 = "";
										$6 = "new"; print $0;
										print NF }'

#Se imprime el número total de registros (6, a-f), se cambia NF a un valor de 3 (a-c), se asigna a la variable 1 el valor de la variable 1 y se imprime toda la línea (que como se ha modificado NF a 3, la línea ya no es abcdef, si no solo abc).
echo a b c d e f | awk '{ print "NF =", NF; NF = 3;
											$1 = $1; print $0 }'
\end{lstlisting}