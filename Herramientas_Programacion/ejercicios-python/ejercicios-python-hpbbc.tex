\documentclass{config/ejercicios}

\title{Ejercicios de Python (HPBBC)}
\author{Sandra Mingo Ramírez}
\date{2024/25}

\usepackage[all]{nowidow}
\usepackage{listing}
\usepackage{color}
\usepackage{tabularx}

\definecolor{dkgreen}{rgb}{0,0.6,0}
\definecolor{gray}{rgb}{0.5,0.5,0.5}
\definecolor{mauve}{rgb}{0.58,0,0.82}

\lstset{frame=tb,
  language=Python,
  aboveskip=3mm,
  belowskip=3mm,
  showstringspaces=false,
  columns=flexible,
  basicstyle={\small\ttfamily},
  numbers=none,
  numberstyle=\tiny\color{gray},
  keywordstyle=\color{blue},
  commentstyle=\color{dkgreen},
  stringstyle=\color{mauve},
  breaklines=true,
  breakatwhitespace=true,
  tabsize=3
}

\begin{document}

\maketitle

\begin{problemS} \textbf{Tabla de multiplicar.}
Escribe un programa que muestre la tabla de multiplicar de un número que introduzca el usuario.
\begin{lstlisting}
#Pide al usuario que introduzca un número por teclado
comprobacion = False
while comprobacion == False:
	try:
		num_usuario = int(input("Introduzca un número entero: "))
	except ValueError:
		print("El valor introducido no es un número entero. Pruebe otra vez.").
	else:
		comprobacion = True
#Utiliza un bucle for para mostrar la tabla de multiplicar
for i in range(1, 11):
	print(f"{num_usuario} x {i} = {num_usuario * i}")
	
#Bonus: haz el mismo ejercicio usando un bucle while
num_multiplicacion = 1

while num_multiplicacion < 10:
	print(f"{num_usuario} x {i} = {num_usuario * i}")
	num_multiplicacion += 1
\end{lstlisting}
\end{problemS}

\begin{problemS} \textbf{Suma de los primeros N números.}
Escribe un programa que pida al usuario un número entero positivo nullN y luego calcule la suma de todos los números desde 1 hasta nullN.
\begin{lstlisting}
#Pide al usuario que introduzca un número entero positivo
comprobacion = False
while comprobacion == False:
	try:
		num_usuario = int(input("Introduzca un número entero positivo: "))
	except ValueError:
		print("El valor introducido no es un número entero. Pruebe otra vez.").
		num_usuario = int(input("Introduzca un número entero: "))
	else:
		if num_usuario < 0:
			print("El número no es positivo.")
		else:
			comprobacion = True
resultado = 0
for i in range(1, num_usuario +1):
	resultado += 1
#Muestra el resultado de la suma
print(f"El resultado de la suma es {resultado}.")
\end{lstlisting}
\end{problemS}

\begin{problemS} \textbf{Contar nucleótidos en una secuencia de ADN.}
Escribe un programa que pida al usuario una secuencia de ADN y cuente cuántas veces aparece cada uno de los nucleótidos (A, T, C, G).
\begin{lstlisting}
#Pide al usuario que ingrese una secuencia de ADN
secuencia_usuario = input("Ingrese una secuencia de ADN: ")
secuencia = secuencia_usuario.upper()
#Utiliza un bucle para recorrer la secuencia y contar la cantidad de veces que aparece cada nucleótido.
num_A = 0
num_T = 0
num_C = 0
num_G = 0
for base in secuencia:
	if base == "A":
		num_A += 1
	elif base == "T":
		num_T += 1
	elif base == "C":
		num_C += 1
	elif base == "G":
		num_G += 1
	else:
		print("Nucleótido no reconocido")
		break
#Muestra el conteo de cada nucleótido
print(f"Conteo de nucleótidos: \n A: {num_A} \n T: {num_T} \n C: {num_C} \n G: {num_G}")
\end{lstlisting}
\end{problemS}

\begin{problemS} \textbf{Transcripción de ADN a ARN}
Escribe un programa que realice la transcripción de una secuencia de ADN a ARN. En una secuencia de ARN, la base nitrogenada Timina (T) se reemplaza por Uracilo (U).
\begin{lstlisting}
#Pide al usuario que ingrese una secuencia de ADN
secuencia_adn = input("Ingrese una secuencia de ADN: ")
#Utiliza un bucle para recorrer la secuencia y reemplazar todas las apariciones de T por U
bases_adn = "ATCG"
secuencia_arn = ""
for base in secuencia_adn:
	if base not in bases_adn:
		print("Secuencia no válida.")
		break
	if base == "T":
		secuencia_arn += "U"
	else:
		secuencia_arn += base
#Muestra la secuencia transcrita de ARN
print(f"Secuencia de ARN transcrita: {secuencia_arn}")
\end{lstlisting}
\end{problemS}

\begin{problemS} \textbf{Devolución de cambio}
Realiza un programa que proporcione el desglose en billetes y monedas de una cantidad entera de euros. Recuerda que hay billetes de 500, 200, 100, 50, 20, 10 y 5 € y monedas de 2 y 1 €.
\begin{lstlisting}
euros_posibles = [500, 200, 100, 50, 20, 10, 5, 2, 1]
cambio = []
cantidad_devolver = int(input("Proporcione la cantidad a devolver: "))
idx = 0 
while cantidad_devolver > 0:
	if cantidad_devolver - euros_posibles[idx] >= 0:
		cambio.append(euros_posibles[idx])
		cantidad_devolver -= euros_posibles[idx]
	else:
		idx += 1
print(cambio)
\end{lstlisting}
\end{problemS}

\begin{problemS} \textbf{Intersecciones de listas}
Diseña una función que reciba dos listas y devuelva los elementos comunes a ambas, sin repetir ninguno (intersección de conjuntos). Ejemplo: si recibe las listas [1, 2, 1] y [2, 3, 2, 4], debe devolver la lista [2]. Escribe, también, otra función que reciba dos listas y devuelva los elementos que pertenecen a una o a otra, pero sin repetir ninguno (unión de conjuntos). Ejemplo: si recibe las listas [1, 2, 1] y [2, 3, 2, 4], devolverá [1, 2, 3, 4].
\begin{lstlisting}
def elementos_comunes_listas(lista1, lista2):
	elementos_comunes = []
		for elemento in lista1:
			if elemento in lista2:
				elementos_comunes.append(elemento)
	return elementos_comunes
	
def union_elementos_listas(lista1, lista2): #No es eficiente al realizar dos bucles for.
	 elementos_unidos = []
	 for elemento in lista1:
	 	if elemento not in elementos_unidos:
	 		elementos_unidos.append(elemento)
	 for elemento in lista2:
	 	if elemento not in elementos_unidos:
	 		elementos_unidos.append(elemento)
	 return elementos_unidos

def union_elementos_listas_eficiente(lista1, lista2):
	 return list(set(lista1) | set(lista2))
\end{lstlisting}
\end{problemS}

\end{document}
