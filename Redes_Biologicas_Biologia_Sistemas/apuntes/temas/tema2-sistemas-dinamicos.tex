%05/02 - Raúl Guantes
\chapter{Sistemas dinámicos no lineales}
\section{Crecimiento logístico}
El crecimiento de las poblaciones está limitado por los recursos que están disponibles. Para ello, se debe añadir un parámetro nuevo, denominado como $k$, que indica la cantidad de recursos. Suponemos que la población tiene una tasa de crecimiento $r$ positiva. Además, contamos con $N(t)$, que es el número de células a un tiempo t. 

Suponiendo que hay una ecuación diferencial que dice cómo cambia la población con el tiempo en función de la variable y los parámetros r y k:
$$\frac{dN}{dt} = f(N; r, k)$$
Suponiendo que los recursos son ilimitados, $r N$. No obstante, si los recursos son limitados, entonces el tamaño depende de los recursos. Si la población es mayor que la disponibilidad de los recursos, la población debe disminuir y habrá células que mueran. SI el tamaño de la población es pequeño, hay recursos suficientes y la población crece. En otras palabras, la derivada es positiva o negariva según si la población crece o disminuye respectivamente. 
$$r N (1 - \frac{N}{k}$$
Esto se conoce como la ecuación de crecimiento logística. Esta ecuuación es no lineal porque N termina estando al cuadrado. 

$$N(t) = \frac{N(0) k}{N(0) + (k - N(0))e^{-rt}}$$
A tiempo infinito, la población tiende al valor de k, quedando en el equilibrio.
En estado de equilibrio, la derivada en función del tiempo da 0. 
$$0 = r N (1 - \frac{N}{K}) \begin{cases} 
N^1_{eq} = K \rightarrow \text{Estable} \\
N^2_{eq} = 0 \rightarrow \text{Inestable}
\end{cases}
$$

Para entender el comportamiento a tiempos largos (en equilibrio), hay que calcular puntos de equilibrio y la estabilidad. Si tenemos un sistema biológico de verdad, al medir a tiempos largos, siempre va a estar en un estado de equilibrio estable, al ser robustos a fluctuaciones. La estabilidad se puede deducir sin resolver la ecuación diferencial mediante dos formas: la forma gráfica y la forma matemática. 

\subsection{Cálculo de estabilidad de forma gráfica}
Con una sola variable, se pinta el denominado \textbf{plano de fases}. En el eje x está la variable, y en el y la derivada de la variable con respecto al tiempo. 
$$\frac{dN}{dt} = f(N) = rN - \frac{rN^2}{K}$$ 
Los puntos de corte son en 0 y en k, quedando una parábola invertida. Para calcular el máximo hay que poner la derivada en 0.
$$\frac{df}{dN} = 0 = r - \frac{2rN}{k} \rightarrow N = \rightarrow N = \frac{rk}{2r} = \frac{k}{2}$$

El punto de equilibrio en k es estable, porque N siempre va a converger a ese punto, ya que en un lado la función es positiva y en el otro lado negativa.

\subsection{Cálculo de estabilidad de forma matemática o por linealización}
La ventaja de este método es que se puede aplicar al número de variables que se quiera. Esta estabilidad se obtiene por linealización. 
Se busca analizar la función en un valor de equilibrio con una pequeña perturbación. 
$$N = N_{eq} + \Delta N$$
Al derivar:
$$\frac{d(N_{eq} + \Delta N)}{dt} = f(N_{eq} + \Delta N$$
$$(\frac{dN_{eq}}{dt} = 0) + \frac{d \Delta N}{dt}$$

$f(N)$ se aproxima por un polinomio según la serie de Taylor. En otras palabras, se aproxima:
$$f(N) \approx a + bN + cN^2 + dN^3 + \ldots$$ 
Vamos a usar eso para realizar una aproximación alrededor de un punto $N_0$. Para aproximar en una línea recta, debe pasar por ese punto y su pendiente debe ser la pendiente de la función, es decir, la derivada de la función y la variable.
$$a = f(N_0); b = \frac{df}{dN}; f = a + bN$$

$$f(N) \approx f(N_0) + \frac{df(N_0)}{dN} (N-N_0) + \frac{1}{2} \frac{d^2f(N_0)}{dN^2} (N-N_0)^2 + \ldots \frac{1}{n!} \frac{d^n(N_0)}{dN^n}(N-N_0)^n$$
Esta aproximación de Taylor se utiliza para aproximar la función $f(N_{eq} + \Delta N$ alrededor del punto $N_{eq}$. 
$$\frac{d \Delta N}{dt} = f(N_{eq} + \frac{df(N_{eq})}{dN} (N-N_{eq}) \rightarrow \frac{\Delta N}{dt} = f'(N_{eq} \Delta N$$

Si $f'(N_{eq}) > 0$, el punto de equilibrio es inestable, ya que aumentaría en el tiempo ($\Delta N(t) = \Delta N(0) \cdot e^{f'(N_{eq}) t}$). Si $f'(N_{eq}) < 0$, entonces el punto de equilibrio es estable.

